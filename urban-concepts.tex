% %This is a very basic article template. %There is just one section and two
% subsections. \documentclass{article}
\documentclass[twocolumn, 10pt, a4paper]{article}



\usepackage{graphics}
% make hyperrefs
\usepackage{hyperref}
\hypersetup{
    colorlinks,%
    citecolor=black,%
    filecolor=black,%
    linkcolor=black,%
    urlcolor=black
}


% use specific settings to get refs after each section
\usepackage[sectionbib]{natbib}
\usepackage{chapterbib}


\newcommand{\authorfont}{\fontsize{12}{14}\selectfont}
\newcommand{\titlefont}{\fontsize{12}{14}\selectfont \bf}

\begin{document}

% \onehalfspacing

\date{}
\title{\vspace{-9mm} \titlefont Urban Metabolism: Definitions, Concepts and
Methodologies } \author{ \authorfont{Andr\'{e} Ribeiro}\\
\authorfont{\normalsize \emph{Sustainable Energy Systems, MIT Portugal}}\\ 
\authorfont{\normalsize \emph{Instituto Superior T\'{e}cnico, Lisboa,
Portugal}}\\ \authorfont{\normalsize \emph{}}\\
%\authorfont{}}\\
%\authorfont{}\\
\authorfont{~}
\\ % used to add blank lines
\vspace{-14mm}
} \maketitle

\section{ABSTRACT}
some abstract goes here


\tableofcontents


\section{Introduction}

Urban metabolism as a metaphore has been around for quiet a long time now \citep{Wolman1965}   
and the research around it has evolved towards some specific methodologies for urban areas analysis.
The majority of the analysis in the area of Urban Metabolism today is 
linked to \emph{Material Flow Analysis (or Accounting) (MFA)}  
essentialy as a tool compare regions performances in what regards to their demands, waste discharges or emissions. 
One of the many uses of MFA studies can be the production of input data for formal predictive models.

Another important concept that is usually linked to urban metabolism is the self sustainability appraisal.
Self sustainabilty is a broad concept which is largely translated (i.e. reduced) to self suficiency, 
and should be gather more eforts for a formal definition. For this work, \emph{resilience} is the key concept
to assess the (self) sustainabilty of urban areas.  

In this work I'll be focused on defining the conce

These models 
In this paper i'll propose some definitions, concepts and methodologies to 
 the citys' metabolism (analysis) should be put to its own rescue 



However not many authors have settled their opinions  



\section{Definitions}


\subsection{City}


Plain text!!


	


\section{Concepts}

\subsection{Urban Services}

More plain text. 

\section{Methodologies}


\subsection{Measuring Resilience}

\subsection{System Dynamics}



\section{High resolution Input Output coeficients}


For the kind of resilience measurements at aim in this work arises the need for
input output in very small regions (i.e. civil parishes \textit{(port. freguesia)} ) 

The construction of input output matrices for very small regions or even
individual entities is a disagregation exercise which has to be performed with
the proper caution since a lot of assumptions 



\bibliographystyle{chicago}
\bibliography{library}


\end{document}