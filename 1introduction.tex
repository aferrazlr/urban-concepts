\section{Introduction}
Security is classicaly one of the intagible assets that sovereign states are
most keen to have. The Institute for Security and Open Methodologies defines
security as ``a form of protection where a separation is created between the
assets and the threat'' \cite{ISECOM2010}. However, security has been evolving
from the militar-centric notion to a wide scope of areas where a society can
suffer damage, e.g. economic security, energy security or environmental
security. Along with security concerns comes the interest in preventive concepts
such as resilience. 

As \cite{Obrist2010} points out, ``like the notion of sustainability,
`resilience' invokes a positive and prospective view. It focuses on people’s and/or system’s
capacities to cope with, recover from and adapt to various risks and
adversities, and directs attention to the ways in which the state and the civil
society can enhance or erode these capacities. Yet, our knowledge of what
constitutes resilience, how it can be described and analysed, and especially,
how it can be fostered is still limited.''

Certainly from the urban planner perspective, resilience is ultimately
linked with the well being provided to dwellers in the face of a socio-economic
disruptive event (either with positive or negative consequences). In this work
we aimed at providing different measures of resilience both from a static
(input-output analysis) and a dynamic models (system dynamics).

Resilience in social-ecological systems has been studied for decades in several
approaches from more ecological-centered \citep{Holling1973} to more social
(even antropological) centered \citep{Bourdieu1995,Bourdieu1986}.

However, in this work we attemp to approach resilience from the urban metabolism
\citep{Wolman1965} point of view, this is, to assess the resistance of energy
and materials stocks and flows against stocastic events. \emph{The Economics of
Ecosystems and Biodiversity} iniciative states in its lastest report
\citep{TEEB2010} that ``maintaining stocks of natural capital allow the
sustained provision of future flows of ecosystem services, and thereby help to
ensure enduring human well-being.'' and follows:

\begin{quote}
Sustaining these flows also requires a good understanding of how
ecosystems function and provide services, and how they are likely to be
affected by various pressures. Insights from the natural sciences are essential
to understanding the links between biodiversity and the supply of ecosystem
services, including ecosystem resilience – i.e. their capacity to continue to
provide services under changing conditions
\end{quote}





Urban metabolism as a metaphore has been around for quiet a long time now    
and the research around it has evolved towards some specific methodologies for urban areas analysis.
The majority of the analysis in the area of Urban Metabolism today is 
linked to \emph{Material Flow Analysis (or Accounting) (MFA)}  
essentialy as a tool compare regions performances in what regards to their demands, waste discharges or emissions. 
One of the many uses of MFA studies can be the production of input data for formal predictive models.

Another important concept that is usually linked to urban metabolism is the self sustainability appraisal.
Self sustainabilty is a broad concept which is largely translated (i.e. reduced) to self suficiency, 
and should be gather more eforts for a formal definition. For this work, \emph{resilience} is the key concept
to assess the (self) sustainabilty of urban areas.  

In this work I'll be focused on defining the conce

These models 
In this paper i'll propose some definitions, concepts and methodologies to 
 the citys' metabolism (analysis) should be put to its own rescue 



However not many authors have settled their opinions  
